%Chiamata classi e impostazioni fuori documento. 

\documentclass{beamer}
\usepackage[italian]{babel} 
\usepackage[latin1]{inputenc} 
\usepackage[T1]{fontenc}

\title{Esempio di titolo} 
\author{Nome autore \\ \texttt{riferimento autore}} 
%\date[VII SINP]{VII Simposio Internazionale sui Numeri Primi} 
\institute{Nome istituto} 
%\logo{\includegraphics[width=15mm]{sigillo}}

\usetheme{AnnArbor} 
\useoutertheme[right]{sidebar} 
\setbeamercovered{dynamic}

%inizio socumento
\begin{document}


%Prima slide (Copertina)
	\begin{frame} 
		\maketitle 
	\end{frame}

%seconda slide
	\begin{frame} 
		\frametitle{Titolo della slide (Frame)} 
		\tableofcontents 
	\end{frame}

%Primo frame complesso
	\section{Esempio di sezione} 
	\begin{frame} 
		\frametitle{Tirolo} 
		\framesubtitle{Sottotitolo} 

%questa parte si potrebbe formattare con le section
			Prova di scrittura di qualcosa.

	\pause 
	
		\begin{enumerate}
			\item Primo elemento. 
			\item Secondo elemento.
			\item Terzo elemento.
			\qedhere 
		\end{enumerate} 

\end{frame}




\end{document}




